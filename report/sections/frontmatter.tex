\vspace*{8em}
\newcommand\tpbreak{\\[\baselineskip]}
\thispagestyle{empty}

\begin{center}
    THE ROLE OF MACHINE LEARNING IN\\ SARS-CoV-2 SUSCEPTIBILITY CLASSIFICATION
 \tpbreak
    by \tpbreak
    Zak White \\
    University of Victoria \tpbreak
    A graduating thesis submitted in partial fulfilment of the requirements for the Honours degree of \tpbreak
    BACHELOR OF SCIENCE \tpbreak
    in the Department of Computer Science
    
    \vfill
    
    \textcopyright\xspace 2022 Zak White \\
    University of Victoria \tpbreak    
    All rights reserved. This thesis may not be reproduced in whole or in part, by photocopying or other means, without the permission of the author. \tpbreak
    We acknowledge with respect the Lekwungen peoples on whose traditional territory the university stands and the Songhees, Esquimalt, and WSANEC peoples whose historical relationships with the land continue to this day.
\end{center}


\newpage

\vspace*{8em}
\begin{center}
    THE ROLE OF MACHINE LEARNING IN\\ SARS-CoV-2 SUSCEPTIBILITY CLASSIFICATION \tpbreak
    by \tpbreak
    Zak White \\
    University of Victoria
\end{center}

% \vfill

\section*{Abstract}
\addcontentsline{toc}{section}{Abstract}

SARS-CoV-2 is a contagious virus established to affect not only humans, but other mammal species. Studies over the last two years have revealed certain species are distinctly immune to the virus, which can be attributed to differences in the ACE2 protein, the virus' target protein, in various hosts. This study applies machine learning methods to classify hosts as susceptible or immune to SARS-CoV-2 based on their ACE2 sequences.

\vspace{1em}\noindent
Machine learning has faced criticism within the field of biology for its uninterpretable logic; it's imperative that biologists and medical professionals can have confidence in the tools they use, and this isn't always possible with machine learning. This is an investigation into the importance of explainable machine learning within bioinformatics that involves the comparison of three models with varying degrees of biological considerations.

\vspace{1em}\noindent
This study validated the hypothesis that biology-driven machine learning applications outperform pure machine learning models, and produced other interesting findings on how mutations in the ACE2 protein can affect susceptibility to SARS-CoV-2.


\newpage
\tableofcontents
\addcontentsline{toc}{section}{Table of Contents}

\newpage
\listofcombined
\addcontentsline{toc}{section}{List of Tables and Figures}

\newpage
\section*{Acknowledgments}
\addcontentsline{toc}{section}{Acknowledgments}

I would like to thank:

My thesis supervisor \textbf{Dr. Hosna Jabbari} for offering me the opportunity to perform this research, inspiring the project topic, and for her guidance through the process. Further, the members of the \textbf{Computational Biology Research and Analytics (COBRA)} group at the University of Victoria for their resources and feedback.

\textbf{My supportive friends.}

Most profoundly, \textbf{my parents and family} for their endless unconditional support and love at every step of my life leading to this academic achievement.

% Thank you to Dr. Hosna Jabbari (for inspiration, guidance, introduction to concepts, and COBRA)

% Thank you to Tibor and/or Tracey.

% Thank you to my loving friends.

% Thank you to my parents.

\newpage
\section*{Dedication}
\addcontentsline{toc}{section}{Dedication}
\emph{To Isla.}

\vspace{1em}

\noindent\emph{Everything in this world is yours.}

\newpage
\
\newpage