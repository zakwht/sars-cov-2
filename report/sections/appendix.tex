\newpage
\begin{appendices}

\section{Sequence Lists}
\newcommand{\spec}[5]{#1 (\emph{#2}) & \href{https://www.ncbi.nlm.nih.gov/protein/#3}{#3} & #4 #5 }

\newcommand{\cA}{\cite{Luan2020}}
\newcommand{\cB}{\cite{Zhao2020}}
\newcommand{\cC}{\cite{Leroy2020}}
\newcommand{\cD}{\cite{Richard2020}}
\newcommand{\cE}{\cite{OudeMunnink2021}}
\newcommand{\cF}{\cite{Palmer2021}}
\newcommand{\cG}{\cite{Chan2020}}
\newcommand{\cH}{\cite{OIE2022}}
\newcommand{\cI}{\cite{Sreenivasan2020}}
\newcommand{\cJ}{\cite{Shi2020}}
\newcommand{\cK}{\cite{Lu2020}}
\newcommand{\cL}{\cite{Woolsey2020}}
\newcommand{\cM}{\cite{Schlottau2020}}

\enlargethispage{2em}
\begin{table}[ht!]
    \begin{adjustwidth}{-12em}{-12em}
    \centering
    \begin{tabular}{lcc}
        \hline
        \textbf{Species} & \textbf{Accession} & \textbf{Susceptible} \\
        \hline
        \spec{Dromedary}{camelus dromedarius}{XP_031301717.1}{No}{\cA} \\
        \spec{Raccoon}{procyon lotor}{BAE72462.1}{No}{\cA} \\
        \spec{Greater horseshoe bat}{rhinolophus ferrumequinum}{XP_032963186.1}{No}{\cA} \\
        \spec{Brown rat}{rattus norvegicus}{NP_001012006.1}{No}{\cA\cB} \\
        \spec{House mouse}{mus musculus}{NP_001123985.1}{No}{\cA\cB} \\
        \spec{Platypus}{ornithorhynchus anatinus}{XP_001515597.2}{No}{\cA} \\
        \spec{African bush elephant}{loxodonta africana}{XP_023410960.1}{No}{\cA} \\
        \spec{European hedgehog}{erinaceus europaeus}{XP_007538670.1}{No}{\cA} \\
        \spec{Raccoon dog}{nyctereutes procyonoides}{ABW16956.1}{No}{\cA} \\
        \spec{Meerkat}{suricata suricatta}{XP_029786256.1}{No}{\cA} \\
        \spec{Kangaroo rat}{dipodomys ordii}{XP_012887573.1}{No}{\cA} \\
        \spec{Guinea pig}{cavia porcellus}{XP_023417808.1}{No}{\cA} \\
        \spec{Domestic pig}{sus domesticus}{ASK12083.1}{No}{\cI\cJ} \\
        \spec{Mallard}{anas platyrhynchos}{XP_012949915.3}{No}{\cI\cJ} \\
        \spec{Human}{homo sapiens}{BAB40370.1}{Yes}{} \\
        \spec{Rhesus macaque}{macaca mulatta}{XP_028697658.1}{Yes}{\cA\cB} \\
        \spec{European rabbit}{oryctolagus cuniculus}{QHX39726.1}{Yes}{\cB} \\
        \spec{House cat}{felis catus}{XP_044906242.1}{Yes}{\cA\cH\cC\cB} \\
        \spec{Domestic dog}{canis familiaris}{NP_001158732.1}{Yes}{\cA\cH\cC\cB} \\
        \spec{Siberian tiger}{panthera tigris}{XP_042830022.1}{Yes}{\cB} \\
        \spec{Golden snub-nosed monkey}{rhinopithecus roxellana}{XP_010364367.2}{Yes}{\cA} \\
        \spec{Olive baboon}{papio anubis}{XP_021788733.1}{Yes}{\cA} \\
        \spec{Chimpanzee}{pan troglodytes}{XP_016798468.1}{Yes}{\cA} \\
        \spec{Orangutan}{pongo abelii}{XP_024096013.1}{Yes}{\cA} \\
        \spec{Ferret}{mustela putorius furo}{NP_001297119.1}{Yes}{\cH\cE\cJ\cD} \\
        \spec{Mink}{mustela lutreola biedermanni}{QNC68911.1}{Yes}{\cH\cE} \\
        \spec{White-tailed deer}{odocoileus virginianus texanus}{XP_020768965.1}{Yes}{\cH\cF} \\
        \spec{Golden hamster}{mesocricetus auratus}{XP_005074266.1}{Yes}{\cH\cG} \\
        \spec{Canadian lynx}{lynx canadensis}{XP_030160839}{Yes}{\cH} \\
        \spec{North american river otter}{lontra canadensis}{XP_032736029.1}{Yes}{\cH} \\
        \spec{Cougar}{puma concolor}{XP_025790417.1}{Yes}{\cH} \\
        \spec{Western lowland gorilla}{gorilla gorilla gorilla}{XP_018874749.1}{Yes}{\cH} \\
        \spec{Spotted hyena}{crocuta crocuta}{KAF0878287.1}{Yes}{\cH} \\
        \spec{Amur leopard}{panthera pardus orientalis}{XP_019273509.1}{Yes}{\cH} \\
        \spec{Pangolin}{manis pentadactyla}{QLH93383.1}{Yes}{\cA\cI} \\
        \spec{Big-eared horseshoe bat}{rhinolophus macrotis}{ADN93471.1}{Yes}{\cA\cI} \\
        \spec{Leschenault's rousette}{rousettus leschenaultii}{BAF50705.1}{Yes}{\cA\cI} \\
        \spec{Common marmoset}{callithrix jacchus}{XP_008987241.1}{Yes}{\cE\cK} \\
        \spec{Cynomolgus macacque}{macaca fascicularis}{XP_005593094.1}{Yes}{\cE\cK} \\
        \spec{Greater short-nosed fruit bat}{cynopterus sphinx}{QKE49997.1}{Yes}{\cE\cM} \\
        \spec{Green monkey}{chlorocebus sabaeus}{XP_037842285.1}{Yes}{\cL\cE} \\
        \hline
    \end{tabular}
    \end{adjustwidth}
    \figcaption{Table}{Species with confirmed accuracies}
\end{table}
% \vspace{-2cm}
% hamster, deer, ferret, cat and dog (C), greenmonk all have unique sources
% [A=E,K]: https://www.ncbi.nlm.nih.gov/pmc/articles/PMC7434851/
% [B=E,L]: https://www.biorxiv.org/content/10.1101/2020.05.17.100289v1
% [C=E,M]: https://www.ncbi.nlm.nih.gov/pmc/articles/PMC7340389/ (and pigs)
% common marmoset (callithrix jacchus): A
% cyno macaque (macaca fascicularis): A
% GSNF bat (Cynopterus sphinx): C
% green monkey (chlorocebus sabaeus): B

Table A.1 lists the full set of sequences used for testing and training the models, and Table A.2 lists sequences used for susceptibility prediction, each with accession IDs from the NCBI database. 
ABC!

\newcommand{\specpred}[3]{#1 (\emph{#2}) & \href{https://www.ncbi.nlm.nih.gov/protein/#3}{#3} }

\begin{table}[h!]
    \centering
    \begin{tabular}{lc}
        \hline
        \textbf{Species} & \textbf{Accession} \\
        \hline
        \specpred{Chinchilla}{chinchilla lanigera}{XP_013362429.1} \\
        \specpred{Grizzly bear}{ursus arctos horribilis}{XP_026333865.1} \\
        \specpred{Black flying fox}{pteropus alecto}{XP_006911709.1} \\
        \specpred{Sumatran orangutan}{pongo abelii}{XP_024096013.1} \\
        \specpred{Tasmanian devil}{sarcophilus harrisii}{XP_031814825.1} \\
        \specpred{Orca}{orcinus orca}{XP_033283817.1} \\
        \specpred{Turkey}{meleagris gallopavo}{XP_019467554.1} \\
        \specpred{Leatherback sea turtle}{dermochelys coriacea}{XP_043360132.1} \\
        \hline
    \end{tabular}
    \figcaption{Table}{Orthologs used for susceptibility prediction}
\end{table}

Table A.3 lists the protein structure models used for the structural analysis, from the AlphaFold Protein Structure Database. Structure \href{https://www.rcsb.org/structure/6VW1}{6VW1} was also used to model the binding between the virus spike protein and ACE2, retrieved from the RCSB Protein Data Bank.

\newcommand{\specstruc}[3]{#1 (\emph{#2}) & \href{https://alphafold.ebi.ac.uk/entry/#3}{#3} }

\begin{table}[h!]
    \begin{adjustwidth}{-12em}{-12em}
    \centering
    \begin{tabular}{lc}
        \hline
        \textbf{Species} & \textbf{Accession} \\
        \hline
        \specstruc{Human}{homo sapiens}{Q9BYF1} \\
        \specstruc{House mouse}{mus musculus}{Q8R0I0} \\
        \specstruc{Brown rat}{rattus norvegicus}{Q5EGZ1} \\
        \hline
    \end{tabular}
    \end{adjustwidth}
    \figcaption{Table}{Orthologs used for structure analysis}
\end{table}

\newpage
\section{Structural Analysis}
Table B.1 summarizes the findings of the structural analysis, noting whether or not each position appeared as a potential binding site, and whether or not the mutation affected the protein's structure.
\begin{table}[ht]
  \centering
  \begin{tabular}[t]{ c | p{28em} }
    \hline
    \textbf{Pos} & \textbf{Results of analysis} \\
    \hline
    31 & Identified as a potential binding site of the spike protein. \\
    41 & No sequences in this study include a mutation at this position: rejected from the model (no effect). \\
    66 & Does not appear to be a binding site, mutation has no effect on the structure: rejected from model. \\
    83 & Identified as a potential binding site of the spike protein, mutation has no immediate effect on structure. \\
    113 & Affects bonds with serine at position 105. \\ 
    353 & Identified as a potential binding site of the spike protein, mutation affects bond with aspartate at position 38. \\
    426 & Identified as a potential binding site of the spike protein, mutation affects bonds with acids at positions 425 and 427. \\
    679 & Does not appear to be a binding site, mutation has no effect on the structure: rejected from model. \\
    \hline
  \end{tabular}
  \figcaption{Table}{Summary of structural analysis}
\end{table}

% 31:
% 41: removed - none
% 66: removed - nothing
% 83: binding
% 113: structure
% 353: structure, binding
% 426: structure, binding?
% 679: removed - nothing

\newpage
\section{Calculations}
Below are the formulas for sensitivity (true positive rate, TPR), specificity (true negative rate, TNR), and classification accuracy (ACC).

\begin{equation}
    TPR = \frac{TP}{P} = \frac{TP}{TP + FN}
\end{equation}
\begin{equation}
    TNR = \frac{TN}{N} = \frac{TN}{TN + FP}
\end{equation}
\begin{equation}
    ACC = \frac{TP + TN}{P + N} = \frac{TP + TN}{TP + FN + TN + FP}
\end{equation}
Where:
\begin{itemize}
    \setlength\itemsep{0em}
    \item $P$ is the total number of pos. sequences
    \item $N$ is the total number of neg. sequences
    \item $TP$ is the number of pos. sequences classified as pos. (true positive)
    \item $TN$ is the number of neg. sequences classified as neg. (true negative)
    \item $FP$ is the number of neg. sequences classified as pos. (false positive)
    \item $FN$ is the number of pos. sequences classified as neg. (false negative)
\end{itemize}

\vspace{1em}

Baseline model calculations
\[ TPR = \frac{5}{6} = 0.8333 \]
\[ TNR = \frac{1}{2} = 0.5000 \]
\[ ACC = \frac{5 + 1}{6 + 2} = \frac{6}{8} = 0.7500 \]

\vspace{1em}

Eliminative model calculations
\[ TPR = \frac{14}{17} = 0.8235 \]
\[ TNR = \frac{6}{9} = 0.6667 \]
\[ ACC = \frac{14 + 6}{17 + 9} = \frac{20}{26} = 0.7692 \]

\vspace{1em}

Structural model calculations
\[ TPR = \frac{24}{27} = 0.8889 \]
\[ TNR = \frac{9}{14} = 0.6429 \]
\[ ACC = \frac{24 + 9}{27 + 14} = \frac{33}{41} = 0.8049 \]


\newpage
\section{Statistical Analysis}

Table D.1 summarizes the results of the nine statistical tests performed, including the chi-square test statistic ($\chi^2$), degrees of freedom (DF) and the p-value. No null hypotheses were rejected.

\begin{table}[ht]
  \centering
  \begin{tabular}[t]{ c c c c c }
    \hline
    \multicolumn{5}{c}{\textbf{McNemar's Test}} \\
    \multicolumn{5}{l}{$H_0$: models have equal error rates} \\
    \hline
    \multicolumn{2}{c}{Models} & {\footnotesize $\chi^2$} & DF & p-value \\
    B & E & 0 & 1 & 1 \\
    B & S & 0.125 & 1 & 0.7237 \\
    E & S & 0 & 1 & 1 \\
    \hline
    \multicolumn{5}{c}{\textbf{Proportion Test: Sensitivity}} \\
    \multicolumn{5}{l}{$H_0$: models have equal sensitivities} \\
    \hline
    \multicolumn{2}{c}{Models} & {\footnotesize $\chi^2$} & DF & p-value \\
    B & E & 0 & 1 & 1 \\
    B & S & 0 & 1 & 1 \\
    E & S & 0.0269 & 1 & 0.8697 \\
    \hline
    \multicolumn{5}{c}{\textbf{Proportion Test: Specificity}} \\
    \multicolumn{5}{l}{$H_0$: models have equal specificities} \\
    \hline
    \multicolumn{2}{c}{Models} & {\footnotesize $\chi^2$} & DF & p-value \\
    B & E & 0 & 1 & 1 \\
    B & S & 0 & 1 & 1 \\
    E & S & 0 & 1 & 1 \\
    \hline
  \end{tabular}
  \figcaption{Table}{Summary of performed hypothesis tests}
\end{table}

The following listings are a selection of the R output of the tests, to convey more information including contingency tables for the McNemar tests and confidence intervals for the proportion tests.

\begin{mdframed}[backgroundcolor=black!5,hidealllines=true,innerleftmargin=0.2cm,innerrightmargin=0.2cm,innertopmargin=0.2cm,innerbottommargin=0.2cm]
\begin{verbatim}
> mcnemar.print("baseline", "eliminative", c(23,4,5,9))
        eliminative
baseline  + -
        + 23 5
        -  4 9

  McNemar's Chi-squared test with continuity correction
McNemar's chi-squared = 0, df = 1, p-value = 1

> mcnemar.print("baseline", "structural", c(24,3,5,9))
        structural
baseline  + -
        + 24 5
        -  3 9

  McNemar's Chi-squared test with continuity correction
McNemar's chi-squared = 0.125, df = 1, p-value = 0.7237

> mcnemar.print("eliminative", "structural", c(28,0,1,12))
            structural
eliminative  +  -
          + 28  1
          -  0 12

  McNemar's Chi-squared test with continuity correction
McNemar's chi-squared = 0, df = 1, p-value = 1
\end{verbatim}
\end{mdframed}
\figcaption{Listing}{R output of McNemar tests}
\begin{mdframed}[backgroundcolor=black!5,hidealllines=true,innerleftmargin=0.2cm,innerrightmargin=0.2cm,innertopmargin=0.2cm,innerbottommargin=0.2cm,nobreak=false]
\begin{verbatim}
> sens.test(baseline, eliminative)
baseline sensitivity: 0.833333333333333
eliminative sensitivity: 0.823529411764706
    2-sample test for equality of proportions
X-squared = 2.2428e-31, df = 1, p-value = 1
95 percent confidence interval: -0.3489446  0.3685524

> sens.test(baseline, structural)
baseline sensitivity: 0.833333333333333
structural sensitivity: 0.888888888888889
    2-sample test for equality of proportions
X-squared = 3.7369e-31, df = 1, p-value = 1
95 percent confidence interval: -0.4320077  0.3208966

> sens.test(eliminative, structural)
eliminative sensitivity: 0.823529411764706
structural sensitivity: 0.888888888888889
    2-sample test for equality of proportions
X-squared = 0.026908, df = 1, p-value = 0.8697
95 percent confidence interval: -0.3298345  0.1991156

> spec.test(baseline, eliminative)
baseline specificity: 0.5
eliminative specificity: 0.666666666666667
    2-sample test for equality of proportions
X-squared = 3.504e-32, df = 1, p-value = 1
95 percent confidence interval: -1.0000000  0.7583094
\end{verbatim}
\end{mdframed}
\begin{mdframed}[backgroundcolor=black!5,hidealllines=true,innerleftmargin=0.2cm,innerrightmargin=0.2cm,innertopmargin=0.2cm,innerbottommargin=0.2cm]
\begin{verbatim}
> spec.test(baseline, structural)
baseline specificity: 0.5
structural specificity: 0.642857142857143
    2-sample test for equality of proportions
X-squared = 3.0052e-32, df = 1, p-value = 1
95 percent confidence interval: -1.0000000  0.7370075

> spec.test(eliminative, structural)
eliminative specificity: 0.666666666666667
structural specificity: 0.642857142857143
    2-sample test for equality of proportions
X-squared = 1.5498e-32, df = 1, p-value = 1
95 percent confidence interval: -0.3973015  0.4449206
\end{verbatim}
\end{mdframed}
\vspace{-1em}
\figcaption{Listing}{R output of proportion tests}

\end{appendices}

