\section{Conclusion}

When facing high-stakes decision --- a reality of biology --- having confidence in the tools that influence the decision is essential. When using black-box models, the logic is hidden and inherently less trustworthy; but even white-box models, where the logic is revealed, are not immediately trustworthy when not paired by an intuitive explanation.

The problem of using machine learning to classify sequences as susceptible or immune to SARS-CoV-2 is severely limited by the availability of data. Regardless, this analysis showed that it is possible to use existing data to classify sequences somewhat reliably: the eliminative model achieved an accuracy of 77\%, comparable to the maximum accuracy of 80\% of the structural model.

The hypothesis of this study was validated: the structural model marginally outperformed its counterparts. Equally significantly, this study is an example of how machine learning can be used as a starting point for classification models, and that pairing it with biological reasoning can strengthen their performances.


% blackbox bad, sometimes whitebox bad: just because you understand the logic of a model, doesn't mean you can understand why the model does that, and it's important to trust the process involved in high-stakes decision-making
% Hyptothesis met/validated, it's so important for biologists to be confident in the tools theyre working with, high stakes decisions,  This is not a great ML problem tho in general because of limited data, but this demonstrates some validity to the use of ML for sequence classification in general