\documentclass{article}
\usepackage[utf8]{inputenc}
\usepackage[backend=biber,style=ieee]{biblatex} %for bibliography

\addbibresource{references.bib}
\newcommand{\textcitecomma}[1]{(\citeauthor{#1}, \citeyear{#1})}

\title{CSC499 Project Proposal}
\author{Zak White}
\date{January 28, 2022}

\begin{document}

\begin{center}
    {\Large Project Proposal} \\
    CSC499 \\
    
    Zak White \\
    V00899901 \\
    January 28, 2022 \\
\end{center}

\section*{Motivation}

Within the field of biology, machine learning has been able to outperform humans in certain analytical tasks \cite{Kakadiaris18}, but the use of black box models has been criticized for its absence of explainable output \cite{Rudin2019}. This project will explore the caveats of machine learning–based algorithms that don’t offer a logical interpretation of their findings.

This analysis will focus on the results of an existing study \textcitecomma{White21}, an investigation into mutations in ACE2 proteins that affect a host’s susceptibility to SARS-CoV-2. The approach of the study was naive, comprising an artificial intelligence–based algorithm that compared protein sequences without factoring in the biological significance of the findings. 

\section*{Objective}
Existing at the intersection of bioinformatics and machine learning, this analysis aims to answer questions from the perspective of each field:
\begin{itemize}
    \item Bioinformatics: what do the findings of the original project mean, ie, why might the identified residues be influential?
    \item Machine learning: was the original approach too naive without considering the explainability of the results? How effectively can the findings be used as an indicator of a sequence’s susceptibility to the virus?
\end{itemize}

\section*{Method}

\begin{enumerate}
    \item Design a binary classification model based on the findings of the original project, and use it to classify some new sequences as susceptible or immune to the virus.
    \item Investigate the model’s misclassified sequences (especially those that are known to be susceptible but are classified as immune) to recognize the naivety of the original method.
    \item Explore the original findings to better understand why some sites may be influential, and isolate those that are explainably influential.
    \item Construct a new classification model based on the reduced findings and compare with the original.
\end{enumerate}

\printbibliography

\end{document}
